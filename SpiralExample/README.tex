\documentclass{article}
\usepackage{hyperref} 

\title{Spiral Example Guide}
\author{Gabriele Pernici}
\date{May 2024}

\begin{document}
	
\maketitle

\tableofcontents

\section{Introduction}
This document depicts how to use this example to reproduce the data regarding the spiral example. Bear in mind that a terminal capable of supporting the \textit{make} and \textit{g++} commands must be used, moreover this guide is written from a linux user's point of view.


\section{Setup}
The first thing to do is to download the GitHub repository and extracting it from its zipped state. After that, you must use the \textit{make} command from terminal, while being in the repository's first folder. This will produce the executable file named \textit{main} inside the same folder.

\section{Training} \label{sec:train}
In order to start the training the following command must be used, while being in the same folder:

\begin{verbatim}
	./main SpiralExample/ComandFiles/Train.cmd
\end{verbatim}

This will start the training algorithm and save every output in the \textit{Results} folder.

\section{Results files}
In the results folder there will be different files: 
\begin{itemize}
	\item 3 log files: \textit{log.txt} (with every iteration result), \textit{test\_log.txt} (with every test result that is performed every 1000 iterations) and \textit{moves\_log.txt} (with every move and decision taken).
	
	\item The struct files (all the \textit{.json}): they contain a loadable structure of the network after every bit increase in the training process. The \textit{struct.json} file contains the final structure.
	
	\item The results files (all the \textit{.exa}): they contain the actual result classification performed by the network on each input sample. The \textit{res.exa} file contains the final result.
\end{itemize}

\section{Evaluation} \label{sec:eval}
After the training, in order to do the evaluation the following command must be used in the usual folder: 

\begin{verbatim}
	./main SpiralExample/ComandFiles/Eval.cmd
\end{verbatim}

This will produce the results used for the graphical representation of the classification. The results will be placed in the \textit{Results} folder as \textit{eval.exa}.

\section{Python scripts}
There are 4 python scripts in the repository (inside the \textit{PythonScripts} folder). What each of them is used for the following:

\begin{itemize}
	\item \textit{GenerateSamplesForImage.py} is used to generate the input file (\textit{Spiral\_evaluation.exa}) for the evaluation described in section \ref{sec:eval}.
	
	\item \textit{GenerateSamplesForTrain.py} is used to generate the input file (\textit{Spiral\_input.exa}) for the training described in section \ref{sec:train}.
	
	\item \textit{PlotNumericalResults.py} is used to analyze the progress of the training and its generalization (described in the \textit{log.txt} and \textit{test\_log.txt}).
	
	\item \textit{PlotSpiralImage.py} is used to plot the results of the evaluation process (described in section \ref{sec:eval}) and saves the image inside the \textit{Results} folder as \textit{eval.jpg}.
\end{itemize}
	
\end{document}
